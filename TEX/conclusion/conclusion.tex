Simulations of a system consisting of transacting agents was performed and analysed. Three different models allowed for different behaviour of the agents which greatly influenced the final wealth distribution of the population. In a first model we allow the agents to save a percentage of their money, represented by the variable $\lambda$, before comitting to a transaction. A greater value of $\lambda$ resulted in a vastly more fair distribution in the sense that the spread in the distribution was low and most agents had roughly the same amount of money. In the two following models we introduce the two probability factors $\alpha$ and $\gamma$ that influence the chance of two agents performing a transaction. In pracsis this was implemented by first picking a specific pair and then calculating the likelyhood of an interaction between these two. The pair would then skip the interaction if some randomly generated number between 0 and 1 was higher than this probability and vice versa. $\alpha$ represented the degree to which agents took into account the difference in their net worth. When $\alpha$ increased there was, on average, a larger discrepancy between the rich versus the poor. The number of agents whose net worth was lower than roughly half of the average was increased while the number of agents  above this decreased. The $\gamma$ variable represented the tendency for agents to transact with partners they had already transacted with before and this tendency increased with the variable. In the third model $\gamma$ was implemented alongside $\alpha$ contributing to a secondary factor for the probability calculation. A larger $\alpha$ accentuated the effects of the $\gamma$ variable and had otherwise only a small effect to the end wealth distribution. Double logarithmic plots displayed a linearity of the tail ends of the distribution, a strong indication of power laws. Qualitatively the exponent on the power law was negative and its absolute value decreased when $\alpha$ was increased. When we varied $\gamma$ we again witnessed a linearity on the double logarithmic plots and the same trend was found. An increase in $\gamma$ meant a decrease in the absolute value of the power law exponent. 
\\\\
For future work one should try to make the trade ratio time dependent. The denominator in the trade ratio becomes larger over time, but a more realistic model would consider a more dynamical view of trade relations where there are diminishing returns and/or a time dependancy on the relations. The effect of taxing or universal basic income would also be interesting to see, but this was too much work for this project. 