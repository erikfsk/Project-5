\section{Introduction}

Predictions of how the financial marked works has, in the modern age, become a large field of study\cite{publisher}. With the knowledge of how stock exchanges work can a mathematical model be build up to simulate these exchanges in the hope that one can predict prices and trends in the future. These predictions are in turn used as a basis for investments and such. The accuracy and speed of these models are therefore of the utmost importance. When these models predicts the wrong outcome things can go very wrong as it did in 2009\cite{marketcrash}.\\

In this project a simulation is done on a simple closed economy model. The economy model consists of a group of financial agents with the ability to transfer money between themselves, using the Monte Carlo method. The simulation will be compared to the well established Pareto's law \cite{paretoslaw}, and the analysis follows two papers. Patriarca with collaborates made one of the articles.\cite{patriarca} The second one was written by Goswami and Sen. \cite{goswami}

In the first simulation all financial agents are made equal and there is a $100\%$ probability of a transaction taking place when a trade is suggested. This it the simplest case and the model is then compared with a logarithmic plot to verify the trend. Then complexity is added by introducing saving. The model is further refined by adding a willingness to trade with agents with a comparative wealth to themselves. Lastly a favorable bias towards agents in which a trade has happened before is added. The whole job will be parallelized to speed up the simulations.\\


\todo[inline]{legg til noe mer for å dekke en side?}

















































